\thispagestyle{empty}\section*{\ifthenelse{\boolean{english}}{Abstract}{Zusammenfassung}}

Temporal Graph Neural Networks (TGNNs) have emerged as a capable tool for applications on graphs that evolve over time, like social networks or financial transaction graphs. Despite the black-box nature of these TGNNs, the explainability of TGNNs remains mostly unexplored. Particularly, counterfactual explanations, which establish what changes to the input graph result in an alternative model outcome, have not yet been explored to explain TGNNs.

To address explainability in TGNNs, this thesis introduces two novel counterfactual explanation methods, named \acrfull{greedycf} and \acrfull{cftgnn}. The explanation task is conceptualized as a search problem, seeking alterations to the input graph that change model outcomes.
%in which a perturbation to the dynamic input graph that changes the outcome is sought. 
\acrshort{greedycf} performs a search that greedily traverses the search space, while \acrshort{cftgnn} leverages a search algorithm based on Monte Carlo tree search to find counterfactual explanations effectively. Extensive experiments show that \acrshort{cftgnn} and \acrshort{greedycf} are capable explainers that provide concise and expressive explanations. In particular, the experiments reveal the efficacy of \acrshort{cftgnn}, which outperforms other methods and achieves an up to $145\%$ higher success rate in identifying counterfactual input alterations.


%CoDy is shown to achieve superior performance, uncovering counterfactual input perturbations in up to $8.8$ times the cases compared to another recent explainer for TGNNs.




%Despite the capabilities of these TGNNs, they remain opaque models that offer no insights into their reasoning. 

%An dieser Stelle erfolgt eine knappe Zusammenfassung der vorliegenden Arbeit ([engl.] Abstract), die maximal ca. 200 Worte umfassen sollte. Der Sinn und Zweck dieser Zusammenfassung liegt darin, einem interessierten Leser die Entscheidung zu erleichtern, die vorliegende Arbeit überhaupt zu lesen bzw. vor dem Lesen der Arbeit erst einmal in Erfahrung zu bringen, worum es dabei geht. Also eine knappe, motivierende Hinführung zum Problem und wie Sie es gelöst haben.

%Wenn Sie eine Zusammenfassung schreiben, bedenken Sie, dass diese oft auch alleine publiziert wird, d.h. sie sollte unabhängig vom nachfolgenden explizit dargestellten Inhalt der Arbeit für den Leser verständlich sein. Daher ist es immer sinnvoll, diese Zusammenfassung erst ganz am Ende zu schreiben, wenn Sie die eigentliche Arbeit bereits abgeschlossen haben.